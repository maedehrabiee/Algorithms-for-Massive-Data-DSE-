% Options for packages loaded elsewhere
\PassOptionsToPackage{unicode}{hyperref}
\PassOptionsToPackage{hyphens}{url}
\documentclass[
]{article}
\usepackage{xcolor}
\usepackage{amsmath,amssymb}
\setcounter{secnumdepth}{-\maxdimen} % remove section numbering
\usepackage{iftex}
\ifPDFTeX
  \usepackage[T1]{fontenc}
  \usepackage[utf8]{inputenc}
  \usepackage{textcomp} % provide euro and other symbols
\else % if luatex or xetex
  \usepackage{unicode-math} % this also loads fontspec
  \defaultfontfeatures{Scale=MatchLowercase}
  \defaultfontfeatures[\rmfamily]{Ligatures=TeX,Scale=1}
\fi
\usepackage{lmodern}
\ifPDFTeX\else
  % xetex/luatex font selection
\fi
% Use upquote if available, for straight quotes in verbatim environments
\IfFileExists{upquote.sty}{\usepackage{upquote}}{}
\IfFileExists{microtype.sty}{% use microtype if available
  \usepackage[]{microtype}
  \UseMicrotypeSet[protrusion]{basicmath} % disable protrusion for tt fonts
}{}
\makeatletter
\@ifundefined{KOMAClassName}{% if non-KOMA class
  \IfFileExists{parskip.sty}{%
    \usepackage{parskip}
  }{% else
    \setlength{\parindent}{0pt}
    \setlength{\parskip}{6pt plus 2pt minus 1pt}}
}{% if KOMA class
  \KOMAoptions{parskip=half}}
\makeatother
\usepackage{longtable,booktabs,array}
\newcounter{none} % for unnumbered tables
\usepackage{calc} % for calculating minipage widths
% Correct order of tables after \paragraph or \subparagraph
\usepackage{etoolbox}
\makeatletter
\patchcmd\longtable{\par}{\if@noskipsec\mbox{}\fi\par}{}{}
\makeatother
% Allow footnotes in longtable head/foot
\IfFileExists{footnotehyper.sty}{\usepackage{footnotehyper}}{\usepackage{footnote}}
\makesavenoteenv{longtable}
\usepackage{graphicx}
\makeatletter
\newsavebox\pandoc@box
\newcommand*\pandocbounded[1]{% scales image to fit in text height/width
  \sbox\pandoc@box{#1}%
  \Gscale@div\@tempa{\textheight}{\dimexpr\ht\pandoc@box+\dp\pandoc@box\relax}%
  \Gscale@div\@tempb{\linewidth}{\wd\pandoc@box}%
  \ifdim\@tempb\p@<\@tempa\p@\let\@tempa\@tempb\fi% select the smaller of both
  \ifdim\@tempa\p@<\p@\scalebox{\@tempa}{\usebox\pandoc@box}%
  \else\usebox{\pandoc@box}%
  \fi%
}
% Set default figure placement to htbp
\def\fps@figure{htbp}
\makeatother
\setlength{\emergencystretch}{3em} % prevent overfull lines
\providecommand{\tightlist}{%
  \setlength{\itemsep}{0pt}\setlength{\parskip}{0pt}}
\ifPDFTeX
  \TeXXeTstate=1
  \newcommand{\RL}[1]{\beginR #1\endR}
  \newcommand{\LR}[1]{\beginL #1\endL}
  \newenvironment{RTL}{\beginR}{\endR}
  \newenvironment{LTR}{\beginL}{\endL}
\fi
\ifluatex
  \newcommand{\RL}[1]{\bgroup\textdir TRT#1\egroup}
  \newcommand{\LR}[1]{\bgroup\textdir TLT#1\egroup}
  \newenvironment{RTL}{\textdir TRT\pardir TRT\bodydir TRT}{}
  \newenvironment{LTR}{\textdir TLT\pardir TLT\bodydir TLT}{}
\fi
\usepackage{bookmark}
\IfFileExists{xurl.sty}{\usepackage{xurl}}{} % add URL line breaks if available
\urlstyle{same}
\hypersetup{
  hidelinks,
  pdfcreator={LaTeX via pandoc}}

\author{}
\date{}

\begin{document}

{\def\LTcaptype{none} % do not increment counter
\begin{longtable}[]{@{}
  >{\raggedright\arraybackslash}p{(\linewidth - 0\tabcolsep) * \real{1.0000}}@{}}
\toprule\noalign{}
\endhead
\bottomrule\noalign{}
\endlastfoot
\begin{titlepage}
    \noindent

    \centering
    {\Large \tituloTesis \par}
    \vspace{2cm}
    Similar Amazon Book Reviews\par
    \vspace{1cm}
    \textbf{  Maedeh Rabiee}
    \vspace{2cm}
    
    \textbf{Algorithms for Massive Data Project}\par
    \vspace{1cm}
    \textbf{Data science for Economics(DSE)}\par
    \vspace{1cm}
    \textbf{Professor Malchiodi}\par
    \vspace{1cm}
    \textbf{October 2025}\
    
    \vspace{1cm}
    \textbf{University of Milan}}\
    \vspace{1cm}
    \ciudad \par
    \mesanio
\end{titlepage}
\begin{document}
%\pagenumbering{roman} % Start roman numbering
\clearpage\maketitle
\thispagestyle{empty}
%\begin{center}
 %   \begin{figure}[h]
  %      \centering
   %     \includegraphics[width=10cm]{unimi.png}
        %\caption{Your caption here}
  %      \label{fig:logo}
   % \end{figure}
  %  \large {Similar Amazon Books Reviews \\

    
   % Name: Maedeh Rabiee

%Algorithms for Massive Data Project
%\vspace{1cm}
% Data science for Economics(DSE)
%\vspace{1cm}
%Professor Malchiodi
%\vspace{1cm}
%October 2025
%\vspace{3cm}
%University of Milan}
%\end{center}


%\begin{center}
%    \begin{figure}[h]
%        \centering
%        \includegraphics[width=10cm]{unimi.png}
        %\caption{Your caption here}
%        \label{fig:logo}
%    \end{figure}
%\end{center}

%\includegraphics[width=2.01 in,height=2.1in]{unimi.png} \\
\end{longtable}
}
\newpage

Table of Contents

\hyperref[introduction]{1. Introduction \hyperref[introduction]{3}}

\hyperref[dataset-description]{2. Dataset Description
\hyperref[dataset-description]{4}}

\hyperref[exploratory-data-analysis-eda]{3. Exploratory Data Analysis
(EDA) \hyperref[exploratory-data-analysis-eda]{5}}

\hyperref[missing-values]{3.1  Missing Values
\hyperref[missing-values]{5}}

\hyperref[duplicate-detection]{3.2 Duplicate Detection
\hyperref[duplicate-detection]{5}}

\hyperref[data-organization-and-pre-processing]{4. Data Organization and
Pre-processing \hyperref[data-organization-and-pre-processing]{5}}

\hyperref[sampling]{4.1 Sampling \hyperref[sampling]{5}}

\hyperref[text-normalization]{4.2 Text Normalization
\hyperref[text-normalization]{6}}

\hyperref[shingling]{5. Shingling \hyperref[shingling]{6}}

\hyperref[MinHash Signatures]{6. MinHash Signatures \hyperref[MinHash Signatures]{6}}

\hyperref[locality-sensitive-hashing-lsh]{7. Locality-Sensitive Hashing
(LSH) \hyperref[locality-sensitive-hashing-lsh]{6}}

\hyperref[lsh-parameters-and-setup]{7.1 LSH Parameters and Setup
\hyperref[lsh-parameters-and-setup]{7}}

\hyperref[similarity-computation-and-verification]{8. Similarity
Computation and Verification
\hyperref[similarity-computation-and-verification]{7}}

\hyperref[results]{8.1 Results \hyperref[results]{8}}

\hyperref[similarity-distribution]{8.2 Similarity Distribution
\hyperref[similarity-distribution]{8}}

\hyperref[similarity-evaluation]{9. Similarity Evaluation
\hyperref[similarity-evaluation]{8}}

\hyperref[example-of-similar-pairs]{10. Example of Similar Pairs
\hyperref[example-of-similar-pairs]{9}}

\hyperref[conclusion]{11. Conclusion \hyperref[conclusion]{10}}

\hyperref[references]{12. References \hyperref[references]{10}}

\hyperref[declaration-by-author]{13. Declaration by author
\hyperref[declaration-by-author]{11}}

\newpage
\section{Introduction}\label{introduction}

This project focuses on detecting similar book reviews within a large
textual dataset using Jaccard similarity, MinHash signatures, and
Locality-Sensitive Hashing (LSH). The primary objective is to
efficiently identify near-duplicate or highly similar reviews among
millions of Amazon Book Reviews without performing exhaustive pairwise
similarity computations across all documents.

Traditional similarity computation exhibits a quadratic time complexity
of O(n²), rendering it computationally infeasible for large-scale
datasets. To address this challenge, MinHashing and LSH techniques are
employed to approximate Jaccard similarity and drastically reduce the
number of required comparisons while maintaining reliable accuracy.

For implementation, the Datasketch Python library was utilized to
perform both MinHash signature generation and LSH indexing. Datasketch
offers efficient and well-optimized data structures for probabilistic
similarity estimation, enabling scalable processing of large textual
datasets.

Furthermore, Datasketch performs exceptionally well within the free tier
of Google Colab, which provides limited memory and computational
resources. Its efficient design allows the entire pipeline---from
shingling to similarity detection---to execute smoothly even in
constrained environments, without requiring paid GPU/TPU support. This
makes Datasketch a practical and powerful choice for academic
experiments and large-scale text similarity detection tasks conducted on
cloud-based platforms such as Colab.
\newpage
\section{Dataset Description}\label{dataset-description}

The dataset was obtained from Kaggle:

�� "Amazon Books Reviews" by Mohamed Bakhet

The dataset used in this project contains approximately 3 million Amazon
book reviews, covering 212,404 unique book titles and their
corresponding user reviews. It comprises 10 columns, each providing
distinct metadata about books and reviewers:

{\def\LTcaptype{none} % do not increment counter
\begin{longtable}[]{@{}
  >{\raggedright\arraybackslash}p{(\linewidth - 2\tabcolsep) * \real{0.2661}}
  >{\raggedright\arraybackslash}p{(\linewidth - 2\tabcolsep) * \real{0.3442}}@{}}
\toprule\noalign{}
\begin{minipage}[b]{\linewidth}\raggedright
Column Name
\end{minipage} & \begin{minipage}[b]{\linewidth}\raggedright
Description
\end{minipage} \\
\midrule\noalign{}
\endhead
\bottomrule\noalign{}
\endlastfoot
Id & Unique book identifier \\
Title & Title of the book \\
Price & Price of the book \\
User\_id & Reviewer's unique ID \\
profileName & Reviewer's name \\
review/helpfulness & Ratio of helpful votes \\
review/score & Rating (1--5) \\
review/time & UNIX timestamp \\
review/summary & Short summary \\
\textbf{review/text} & Full text of the review \\

\end{longtable}
}

Overall, the dataset is rich in textual data and provides sufficient
variation for natural language analysis. The review/text column served
as the primary source for the similarity detection task, as it contains
the complete textual content of each review.
\newpage
\section{Exploratory Data Analysis
(EDA)}\label{exploratory-data-analysis-eda}

Before applying the algorithms, I performed an initial Exploratory Data
Analysis (EDA) to understand the structure, quality, and potential
issues in the dataset.

\subsection{ Missing Values}\label{missing-values}

The table below shows the number and percentage of missing values in
each column:

{\def\LTcaptype{none} % do not increment counter
\begin{longtable}[]{@{}
  >{\raggedright\arraybackslash}p{(\linewidth - 4\tabcolsep) * \real{0.2815}}
  >{\raggedright\arraybackslash}p{(\linewidth - 4\tabcolsep) * \real{0.1941}}
  >{\raggedright\arraybackslash}p{(\linewidth - 4\tabcolsep) * \real{0.2226}}@{}}
\toprule\noalign{}
\begin{minipage}[b]{\linewidth}\raggedright
Column
\end{minipage} & \begin{minipage}[b]{\linewidth}\raggedright
Nulls
\end{minipage} & \begin{minipage}[b]{\linewidth}\raggedright
Percent (\%)
\end{minipage} \\
\midrule\noalign{}
\endhead
\bottomrule\noalign{}
\endlastfoot
Id & 0 & 0.00\% \\
Title & 208 & 0.01\% \\
Price & 2,518,829 & 83.96\% \\
User\_id & 561,787 & 18.73\% \\
ProfileName & 561,905 & 18.73\% \\
Review/Helpfulness & 0 & 0.00\% \\
Review/Score & 0 & 0.00\% \\
Review/Time & 0 & 0.00\% \\
Review/Summary & 407 & 0.01\% \\
Review/Text & 8 & 0.00\% \\
Total & 3,643,144 & --- \\
\end{longtable}
}

\subsection{Duplicate Detection}\label{duplicate-detection}

A duplicate check revealed:

\begin{itemize}
\item
  8,774 rows where all columns were identical (≈0.29\% of the dataset).
\end{itemize}

\section{\texorpdfstring{ Data Organization and
Pre-processing}{ Data Organization and Pre-processing}}\label{data-organization-and-pre-processing}

\begin{itemize}
\item
  Firstly Duplicated 8,774 rows where all columns were identical were
  removed to ensure that each review was unique.
\end{itemize}

For similarity analysis, I retained only the two relevant columns: `Id'
and `review/text'.\\
Rows with missing review texts were dropped, leaving 2,991,218 unique
reviews, which were saved in the df\_clean DataFrame.

\subsection{Sampling}\label{sampling}

Due to computational constraints, a 1\% random sample
(\textasciitilde29,912 reviews) was taken for experimentation and saved
as df\_sample. This subset was large enough to test the efficiency of
LSH while maintaining the dataset's diversity.

\subsection{Text Normalization}\label{text-normalization}

Each review was normalized using the following steps, and the cleaned
text was saved in the text norm column:

\begin{itemize}
\item
  Convert text to lowercase
\item
  Remove punctuation and non-alphanumeric characters
\item
  Collapse multiple spaces
\end{itemize}

\section{\texorpdfstring{ Shingling}{ Shingling}}\label{shingling}

The next step was word-level shingling, where each review was divided
into overlapping groups of three consecutive words (k = 3). \RL{}We
chose k = 3 because it provides a balanced trade-off between recall and
precision. Smaller k values (e.g., k=2) increase recall but can merge
semantically different texts, while larger k values (k=4--5) increase
precision but may miss paraphrases.

Then, Stopwords (such as \emph{the}, \emph{and}, \emph{is}) were removed
to reduce noise.

Example:

Text: ``I really enjoyed reading this book''\\
→ Shingles (k=3): \{``really enjoyed reading'', ``enjoyed reading
book''\}

Each document was therefore represented as a set of shingles, capturing
local word patterns and phrase structure. This representation was saved
in the shingles column.

\section{MinHash Signatures}\label{algorithms}



Since directly comparing large sets is computationally expensive, I
applied \textbf{MinHashing} to create compact representations of each
review's shingles. MinHash simulates random permutations to produce
short signatures, where the probability that two signatures match at a
given position equals their Jaccard similarity.

I used 128 permutations per document to balance accuracy and memory
efficiency. The resulting signatures were stored in the minhash column.

\section{Locality-Sensitive Hashing
(LSH)}\label{locality-sensitive-hashing-lsh}

To avoid comparing every pair of the 29,912 reviews, Locality-Sensitive
Hashing (LSH) was applied on top of the MinHash signatures. LSH divides
each signature into bands and hashes them into buckets. Reviews that
fall into the same bucket in at least one band are considered candidates
for being similar.\\
This method drastically reduced the number of required pairwise
comparisons while maintaining high recall for near-duplicate reviews.

\subsection{LSH Parameters and Setup}\label{lsh-parameters-and-setup}

The following parameters were used for Locality-Sensitive Hashing (LSH):

\begin{itemize}
\item
  \textbf{Threshold (t):} 0.6 (minimum Jaccard similarity)
\item
  \textbf{Bands (b):} 18
\item
  \textbf{Rows per band (r):} 7 $\Rightarrow$ (18 $\times$ 7 $\approx$ 128 hash values)

\end{itemize}

A threshold of \textbf{0.6} was chosen as it provides a balanced
trade-off between recall and precision.\\
A lower threshold (e.g., 0.5) would increase the number of candidate
pairs but also the verification cost, whereas higher thresholds would
risk missing potential duplicates.

For \textbf{LSH index construction}, a unique identifier was generated
for each review to enable efficient indexing.\\
Specifically, the identifier was created by concatenating the book\_id
with the corresponding row index (in the format book\_id\_row\_index).
This composite key ensured that each review remained distinct while
maintaining its association with the original book information. The
resulting identifier was stored in the unique\_key column.

To ensure accuracy, self-pairs and duplicate pairs were removed prior to
verification.

This process yielded the following results:

\begin{itemize}
\item
  \textbf{Candidate pairs:} 331
\item
  \textbf{Total possible pairs:} 29,912 × 29,911 / 2 = 447,348,916
\item
  \textbf{Reduction:} 99.9999\% fewer comparisons
\end{itemize}

These results clearly demonstrate how scalable and efficient the LSH
approach is for massive datasets.

\section{Similarity Computation and
Verification}\label{similarity-computation-and-verification}

The Jaccard similarity function was applied to compute the exact
similarity between the shingles of each candidate pair.\\
This function measures the ratio of the intersection to the union of two
sets, defined as:

\[J(A,B) = \frac{\mid A \cap B \mid}{\mid A \cup B \mid}
\]

Only pairs with a Jaccard similarity above 0.6 were retained as verified
duplicates or near-duplicates

\subsection{Results}\label{results}

\begin{itemize}
\item
  331 verified similar pairs
\item
  Mean similarity: 0.994
\item
  Median similarity: 1.000
\item
  Min similarity:0.667
\end{itemize}

These metrics confirm that the LSH candidate selection was highly
precise, with nearly all shortlisted pairs exhibiting strong semantic or
literal overlap.

\subsection{Similarity Distribution}\label{similarity-distribution}

The table below summarizes the distribution of similarity scores among
the 331 verified pairs:

{\def\LTcaptype{none} % do not increment counter
\begin{longtable}[]{@{}
  >{\raggedright\arraybackslash}p{(\linewidth - 4\tabcolsep) * \real{0.1891}}
  >{\centering\arraybackslash}p{(\linewidth - 4\tabcolsep) * \real{0.1654}}
  >{\raggedright\arraybackslash}p{(\linewidth - 4\tabcolsep) * \real{0.2203}}@{}}
\toprule\noalign{}
\begin{minipage}[b]{\linewidth}\raggedright
Range
\end{minipage} & \begin{minipage}[b]{\linewidth}\centering
Count
\end{minipage} & \begin{minipage}[b]{\linewidth}\raggedright
Percentage (\%)
\end{minipage} \\
\midrule\noalign{}
\endhead
\bottomrule\noalign{}
\endlastfoot
0.60 -- 0.65 & 0 & 0.0\% \\
0.65 -- 0.70 & 1 & 0.3\% \\
0.70 -- 0.75 & 3 & 0.9\% \\
0.75 -- 0.80 & 0 & 0.0\% \\
0.80 -- 0.85 & 2 & 0.6\% \\
0.85 -- 0.90 & 0 & 0.0\% \\
0.90 -- 0.95 & 5 & 1.5\% \\
0.95 -- 1.00 & 320 & 96.7\% \\
Total & 331 & 100.0\% \\
\end{longtable}
}

\section{Similarity Evaluation}\label{similarity-evaluation}

The resulting similarity distribution (Table 1) revealed that 96.7\% of
all pairs had a similarity score between 0.95 and 1.00, confirming that
nearly all detected pairs were true near-duplicates or identical
reviews.

This high-precision outcome demonstrates the effectiveness of the chosen
parameters --- 3-word shingles, 128 MinHash permutations, and an LSH
threshold of 0.6 --- in filtering out dissimilar reviews while
successfully retaining genuine duplicates.

All results were exported into multiple files for transparency and
further analysis:

\begin{itemize}
\item
  similar\_pairs.csv -- a compact version containing review identifiers
  and similarity scores.
\item
  similar\_pairs\_FULL\_TEXT.csv -- includes the complete review texts
  for each pair.
\item
  similar\_pairs\_FULL\_REPORT.txt -- a detailed, human-readable summary
  of all matched pairs.
\item
  summary\_statistics.txt -- provides overall statistics and the
  similarity distribution.
\end{itemize}

Together, these outputs formed a complete, scalable pipeline capable of
detecting and verifying near-duplicate reviews within the Amazon Books
Reviews dataset.

\section{Example of Similar Pairs}\label{example-of-similar-pairs}

Example (Similarity = 1.000)

PAIR \#1

Similarity Score: 1.000

Key 1: B000GLN7HQ\_18665

Key 2: B000HWEE7Q\_22204

REVIEW 1 (COMPLETE TEXT):

This would be a good starting place for someone unfamiliar with
Christie\textquotesingle s works. It is one of the Hercule Poirot
detective stories; I hadn\textquotesingle t read any of them before and
therefore can say that it doesn\textquotesingle t seem to have a
negative effect reading this one out of sequence. (It did make me want
to read more of the series though!)The story is set in a small English
village where gossip is the favored pasttime and nothing much of
significance ever really happens; however, the village inhabitants are
stunned when two mysteries unfold at once: one involving blackmail and a
suicide, and the other involving murder. Hercule Poirot, who has just
moved to the village to retire, is irresistibly pulled into the intrigue
and starts to investigate. He is aided by the village doctor, Dr.
Sheppard, who narrates the tale.It\textquotesingle s been a long time
since I was completely absorbed in a mystery, and this delivered that
deliciously maddening desire to have to know- right now-\/- who done it?
It\textquotesingle s difficult to put it down once started, and the
ending was nothing short of nail-biting shock. Christie is a master at
keeping the reader guessing and then delivering an emotionally stunning
wrap-up of her tale. This one kept me on the edge of my seat, and I
can\textquotesingle t wait to read more of her novels.

REVIEW 2 (COMPLETE TEXT):

(Identical to Review 1)

PAIR \#331

Similarity Score: 0.667

Key 1: 0460112872\_19140

Key 2: B000L28LS0\_15176

REVIEW 1 (COMPLETE TEXT):

Easy to get on my Kindle and to transport it everywhere. Rapid download
to all my connected electronics. And you cannot go wrong with so little
money for it. Love those classics.

REVIEW 2 (COMPLETE TEXT):

Easy to get on my Kindle and to transport it everywhere. Rapid download
to all my connected electronics. And you cannot go wrong with free.

\section{\texorpdfstring{Conclusion }{Conclusion }}\label{conclusion}

In this project, I built a scalable end-to-end pipeline to detect
duplicate and near-duplicate reviews within the Amazon Books Reviews
dataset, which contains approximately three million entries.

The approach is simple yet effective. Each review was converted into
3-word shingles (k=3), then compressed using MinHash with 128 random
permutations, and finally indexed with Locality-Sensitive Hashing (LSH)
using a Jaccard similarity threshold of 0.6 to efficiently shortlist
candidate pairs.

When tested on a 1\% random sample (≈29,912 reviews), the total number
of possible review pairs (about 4.47×10⁸) was reduced to just 331
candidates by the LSH step. After applying the exact Jaccard similarity,
all 331 pairs were confirmed to have similarity values above 0.6, proving
that the method achieved high accuracy while drastically reducing
computational cost.

The entire pipeline was implemented using the Datasketch Python library,
which provides simple yet powerful tools for MinHashing and LSH.
Moreover, it runs smoothly on Google Colab's free environment, making it
an ideal solution for large-scale text processing without requiring
high-end hardware.

Overall, this project demonstrated that MinHashing and LSH, when
combined with the Datasketch library, provide a fast, memory-efficient,
and reliable approach for solving large-scale text similarity problems.

\section{References}\label{references}

\begin{enumerate}
\def\labelenumi{\arabic{enumi}.}
\item
  Rajaraman, A., \& Ullman, J. D. (2011). \emph{Mining of Massive
  Datasets}. Cambridge University Press.
\item
  Datasketch Python Library: \url{https://ekzhu.github.io/datasketch}
\end{enumerate}

\section{\texorpdfstring{\textbf{Declaration by
author}}{Declaration by author}}\label{declaration-by-author}

I declare that this material, which I now submit for assessment, is
entirely my own work and has not been taken from the work of others,
save and to the extent that such work has been cited and acknowledged
within the text of my work. I understand that plagiarism, collusion, and
copying are grave and serious offences in the university and accept the
penalties that would be imposed should I engage in plagiarism, collusion
or copying. This assignment, or any part of it, has not been previously
submitted by me or any other person for assessment on this or any other
course of study.

\end{document}
